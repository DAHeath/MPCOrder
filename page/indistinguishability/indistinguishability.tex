\documentclass{article}

\usepackage{
  amsmath,
  amssymb,
  amsthm,
  amsfonts,
  stmaryrd,
  hyperref,
  cleveref,
  url,
  appendix,
  pdfsync,
  graphicx,
  empheq,
  physics,
}

\newtheorem{example}{Example}
\newtheorem{note}{Working Note}
\newtheorem{theorem}{Theorem}
\newtheorem{conjecture}{Conjecture}
\newtheorem{corollary}{Corollary}
\newtheorem{definition}{Definition}
\newtheorem{remark}{Remark}
\newtheorem{lemma}{Lemma}
\newtheorem{invariant}{Invariant}
\newtheorem{notation}{Notation}

\newcommand{\bit}{\{0,1\}}
\newcommand{\indist}{\approx}
\newcommand{\linked}{\diamond}
\newcommand{\lib}{\mathcal{L}}
\newcommand{\adv}{\mathcal{A}}

\newcommand{\word}[1]{\mathsf{#1}}
\newcommand{\many}[1]{\mathit{#1}}
\newcommand{\defn}{\ensuremath{\triangleq}}


\title{Indistinguishability}

\begin{document}

\maketitle

\section{Definition}

\begin{definition}[Indistinguishability]
  Let $\lib_\ell, \lib_r$ denote two libraries with a common interface. We say
  that these libraries are indistinguishable -- written $\lib_\ell \indist \lib_r$ -- if for any polytime
  procedure $\adv$ the following quantity is \href{negligible.html}{negligible} in $\lambda$:
  \begin{align*}
    |\mathrm{Pr}[(\adv \linked \lib_\ell)(1^\lambda) \Rightarrow 1]
    -
    \mathrm{Pr}[(\adv \linked \lib_r)(1^\lambda) \Rightarrow 1]|
  \end{align*}
\end{definition}

\section{Properties}

Indistinguishability forms an
\href{https://en.wikipedia.org/wiki/Equivalence_relation}{equivalence relation}
on libraries.
More precisely, indistinguishability is trivially reflexive and symmetric, and it is also transitive.

\begin{lemma}[Indistinguishability is transitive]
  Let $\lib_0, \lib_1, \lib_2$ be three libraries with a common interface. If
    $\lib_0 \indist \lib_1$ and $\lib_1 \indist \lib_2$,
    then $\lib_0 \indist \lib_2$.
\end{lemma}
\begin{proof}
  By \href{https://en.wikipedia.org/wiki/Triangle_inequality}{triangle inequality}.

  Suppose the conclusion \emph{does not} hold, i.e. $\lib_0 \not \indist \lib_2$.
  Then by definition there exists a polytime procedure $\adv$ that can distinguish $\lib_0$ and $\lib_2$ with non-negligible advantage.
  TODO.
\end{proof}

\end{document}
